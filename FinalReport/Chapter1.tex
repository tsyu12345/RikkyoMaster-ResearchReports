\chapter{はじめに}
本章では,本論文の要旨および構成について述べる.
\section{要旨}
首都直下型地震や南海トラフ地震をはじめとする,大地震の 30 年以内の発生確率が 70\%~
80\%と非常に高くなっていることに加え,近年の豪雨など,将来の大規模災害のリスクが著
しく高まっている現状がある.
このような背景の下,本研究では,津波避難誘導のにおける避難完了率最適化と誘導人員のリスク軽減を目的として,マルチエージェント強化学習を活用した自律飛行型ドローンによる津波避難誘導モデルを提案する.
複数のドローンが協調して避難者を最適な避難所に誘導する方策を強化学習により求め,ゲームAIと3D都市モデルを用いた現実に近いシミュレーション環境による効果検証を実施した.
本提案手法により,津波避難誘導の課題解決に寄与することを目指す.

\section{本稿の構成}
まず,第2章において本稿の内容を理解するのに必要な事前知識,研究背景について述べる.
具体的には,以下の項目について説明する.
\begin{itemize}
  \item 強化学習についての基本説明
  \item マルチエージェントアルゴリズム MA-POCA(MultiAgent POsthumous Credit Assignment)について
  \item 本研究の社会的背景・課題について
\end{itemize}

次の第3章においては,提案手法の説明と本研究の研究方法についての説明を行う.
第4章では,マルチエージェント強化学習エージェントによる,津波避難誘導のシミュレーション実験の結果と考察を行う.
第5章では,実験結果をまとめ,マルチエージェントドローンによる津波避難誘導の実現可能性について論ずる.また,今後の研究の展望を述べる.

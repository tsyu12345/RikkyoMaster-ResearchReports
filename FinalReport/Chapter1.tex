\chapter{はじめに}
本章では,本論文の要旨および構成について述べる.
\section{要旨}
首都直下型地震や南海トラフ地震をはじめとする,大地震の 30 年以内の発生確率が 70\%~
80\%と非常に高くなっていることに加え,近年の豪雨など,将来の大規模災害のリスクが著
しく高まっている現状がある.
\section{本稿の構成}
まず,第2章において本稿の内容を理解するのに必要な事前知識,研究背景について述べる.
具体的には,以下の項目について説明する.
\begin{itemize}
  \item 強化学習についての基本説明
  \item マルチエージェントアルゴリズム MA-POCA(MultiAgent POsthumous Credit Assignment)について
  \item 本研究の社会的背景・課題について
\end{itemize}

次の第3章においては,本研究の研究手法についての説明を行う.
第4章では,マルチエージェント強化学習エージェントによる,津波避難誘導のシミュレーション実験の結果と考察を行う.
第5章では,実験結果をまとめ,マルチエージェントドローンによる津波避難誘導の実現可能性について論ずる.また,今後の研究の展望を述べる.

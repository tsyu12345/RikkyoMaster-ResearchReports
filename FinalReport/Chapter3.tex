\chapter{提案手法}  
本章では、本研究が最終的に目指す津波避難誘導問題への解決策として、マルチエージェント強化学習と自律飛行型ドローンを組み合わせた提案手法について述べる。
また、提案手法が既存研究と異なる点や新規性についても論じる。

\section{概要}  
本研究では、観光地や都市部といった地元住民以外にも多数の人々が屋外に存在する状況を想定している。
これには、日常的に避難訓練を受けていない観光客や土地勘のない訪問者も含まれる。
このような状況を背景に、地震発生後の津波避難という非常に緊急性の高い場面を想定し、避難誘導を行うための手法を検討する。
従来、自治体職員や警察・消防隊員といった人間が担ってきた避難誘導を、自律飛行型ドローンが代替するシステムを構築することを目標とする。

具体的には、マルチエージェント強化学習を活用し、複数のドローンエージェントが協調して行動する能力を学習させることで、刻々と変化する被災地域の状況を動的に認識し、群衆の避難完了率を最大化することを目指す。
また、避難者の位置、避難経路上の障害物、各ドローンの位置などをリアルタイムに反映するデジタルツイン環境を構築し、その環境内で学習済みのエージェントがシミュレーションを通じて最適な誘導方法を実行できるようにする。
さらに、デジタルツイン環境と実機ドローンを連動させることで、現実世界での運用を可能とする動的な避難誘導システムを構築する。

% TODO: ここに手法をまとめた図を貼る

\section{既存研究との新規性}  
研究背景で述べたとおり、既存の研究と本研究の主な相違点および新規性は以下の通りである:
\begin{itemize}
    \item ドローンの防災活用はまだ新しい研究分野であり、本研究はその分野をさらに推進するものである。
    \item 複数のドローンが連携して運用されるシステムに、AIや強化学習モデルを導入する点。
    \item 都市モデルを活用した訓練環境を構築し、デジタルツイン技術を通じて実環境での運用を想定している点。
\end{itemize}

これらの点を通じて、本研究は現実世界での応用可能性を持つ動的な津波避難誘導システムの実現を目指している。

\section{環境構築}
  \subsection{避難所の選定}
  \subsection{避難者の行動フロー}
\section{エージェントモデル}
  \subsection{移動経路の計算方法}
  \subsection{観測情報}
  \subsection{報酬関数と行動評価}
  

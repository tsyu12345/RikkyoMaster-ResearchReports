\chapter{結論}
本章では、本研究で提案したマルチエージェントドローンによる津波避難誘導システムの実現可能性について、実験結果および我が国における災害対策の現状を踏まえてまとめる。また、本研究の成果を基にした今後の展望についても述べる。

\section{実験のまとめ}
本研究は、マルチエージェント強化学習を用いたドローンによる津波避難誘導システムを提案し、その効果を3D都市モデルを活用したシミュレーション実験で検証した。
実験では,津波避難誘導のタスクを\textbf{1.避難者探索と収集}と\textbf{2.避難者群衆の誘導}.の2つのタスクに置けるマルチエージェントモデルを作成しその効果を検証した.
その結果から、以下の点が明らかとなった。

%TODO: 実験結果からルールベースでのドローンエージェントが有効であることを述べる
\paragraph{ルールベースでのドローンエージェントの有効性}
まず、ルールベースでのドローンエージェントが避難者の探索および誘導において一定の有効性を示した点である。
避難者探索のタスクにおいては,都市の道路状況やエージェントの初期位置,エージェントの機数により結果に変動はあるが,
ランダムに行動するドローンエージェントの場合でも比較的短時間で35\%から最大で80\%程度の避難者を発見,エージェントの各個において誘導すべき群衆を形成することができた.
また,群衆の避難所誘導タスクにおいては,避難所までの最短経路を進む避難者のみで行動するケースよりも,ルールベースでの誘導を実施する方が若干ではあるが避難完了を速く行わせることができることが確認された.

ただし,これらのシミュレーション条件にはいくつか課題が残されている.
探索モデルの場合は,避難者の出現位置の前提条件として,環境内の道路上にランダム偏りなく出現するという条件があり,実際の都市空間においては住宅街や大通りなど場所により避難者の人数分布は偏る可能性がある.
避難者の出現分布を偏らせた場合での検証が必要である.
また,避難所誘導モデルにおいては,エージェントの初期位置の違いにより,エピソード毎の最終的な避難完了率にバラつきがある点や,避難所毎の収容人数に偏りを持たせた場合での検証が必要である.
また,避難者の人数規模についても再考が必要である.本研究ではサンプルとして200名前後での各タスク遂行実験を行ったが,実際の各都市における想定される避難者数を考慮したさらに大規模な検証が必要である.

\paragraph{マルチエージェントモデルの課題}
%TODO: マルチエージェントモデルが上手くいななかった理由を述べる→改善点の提案
本研究では,探索と誘導どちらのタスクにおいても,今回作成したマルチエージェントモデルはルールベースでの性能には及ばなかった.
これは,モデルの学習が全体として上手くいかず,エージェントが適切な協調行動を行う方策を学習できなかったためである.
学習が上手くいかなかった要因としては複数考えられるが,モデルの学習過程を分析すると次の様なことが考えられる.
探索タスクにおいては,累積報酬が学習過程で増加せず,最終的に減少傾向にあった点と価値関数の平均損失が学習過程で減少している点が確認された.
また,誘導タスクにおいては、累積報酬は沼津市のケースでは若干の増加傾向を示したものの横須賀市のケースでは学習過程全体で減少傾向を示した.また全体として,価値関数の平均損失の値が大きく,モデルの予測性能が低いことが伺える.
このことから,エージェントが訓練課程で誤った方策を学習してしまった可能性が高く、最終的な結果が良くならなかったものと推察される.

\section{ドローンによる津波避難誘導の実現可能性}
%TODO: ここから現在の研究背景を元に,実現可能性について論じる.
ドローン(無人航空機)の津波避難における活用が既に我が国でも進められており一部の自治体では,運用段階に入っていることを紹介した.
また東日本大震災以後,避難誘導において避難者のみならず,誘導にあたる人員も含めた人的被害を低減することの重要性が再認識されている現状がある.
加えて,新型コロナウイルスによる政府の自粛要請も終わり,観光客数が再び増加傾向にある中,
土地勘のない大勢の観光客も含めた避難者の津波避難行動については,津波避難ビルの配置や収容定員超過等の理由により,現状では十分な避難対策が行われているとは言い難い状況が先行研究で示された.
このような状況において,津波避難誘導の一連のオペレーションをドローンによって代替することは,人的被害を低減する観点からも有効であると考えられる.
本研究における実験結果が示すように,マルチエージェントニューラルネットワークモデルを用いたドローンの避難誘導には課題が残されているが,
ルールベースでのドローンエージェントによる避難誘導は一定の効果が期待でき,先行研究事例も含め,津波避難誘導をドローンで代替できる可能性とその有効性を確認することができた.
\section{今後の展望}
本研究の結果を踏まえ、以下の方向性でさらなる発展が期待される。

まず、報酬設計の見直しが必要である。具体的には、避難者を発見した際の報酬だけでなく、未探索エリアの探索や協調行動に基づく報酬を導入することで、エージェントの探索効率を向上させることが可能である。また、エントロピーの制御を通じて、行動方針の収束を促進しながら適切な探索と利用のバランスを取る仕組みを設ける必要がある。

次に、シミュレーション環境をさらに現実に近づけることも重要である。本研究で構築した環境は、現実の地形や避難所配置を模倣したものであるが、気象条件やリアルタイムの人口動態データを取り入れることで、より現実的なシナリオでの検証が可能となるであろう。

さらに、実環境における検証も進めるべきである。例えば、自治体や研究機関と連携し、災害対応訓練の一環として提案システムを試験運用することで、その有効性を実証し、改良を行うことができる。また、複数ドローンのリアルタイム制御や通信の安定性を向上させるための技術的な工夫も求められる。

最終的に、本研究の成果は、津波避難誘導に限らず、風水害や地震など、様々な災害シナリオへの応用が期待される。マルチエージェント強化学習と自律型ドローンの組み合わせは、防災技術の新たな可能性を広げるものであり、今後の防災分野におけるさらなる研究と実用化が待たれる。

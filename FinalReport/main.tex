\documentclass[a4paper, 12pt]{bxjsreport}
\usepackage{fontspec}
\usepackage{zxjatype}
\usepackage{amsmath} % 数式コマンドの拡張
\usepackage{amssymb} % 追加の数学記号
\usepackage{amsfonts} % 数学フォント(\mathbb 用)
\usepackage{siunitx} % SI単位系
\usepackage{tabularx}
\usepackage{autobreak}
\usepackage{bm}
\usepackage[dvipdfmx]{graphicx}
\usepackage{float}
\usepackage{cite}
\usepackage{enumerate}
\usepackage[notransparent]{svg}
\usepackage{graphicx}   % 画像を扱うパッケージ
\usepackage{subcaption} % サブキャプションを扱うパッケージ

\geometry{left=25truemm, top=10truemm, right=25truemm, bottom=20truemm}
\setjamainfont{ipam.ttf}
\setjasansfont{ipag.ttf}

\makeatletter
\newcommand{\biggg}[1]{{\hbox{$\left#1\vbox to 20.5pt{}\right.\n@space$}}}
\newcommand{\Biggg}[1]{{\hbox{$\left#1\vbox to 23.5pt{}\right.\n@space$}}}
\newcommand{\ctext}[1]{\raise0.2ex\hbox{\textcircled{\scriptsize{#1}}}}

\begin{document}

\begin{titlepage}
    \begin{center}

        \includesvg[width=110pt]{RIKKYOAI_main.svg}

        \vspace{50truept}

        %%%%%%%%% 修論は以下 %%%%%%%%%
        % {\headfont \Large 2024年度 修士論文}
        % \vspace{20truept}

        % {\headfont \Huge タイトル}
        % \vspace{160truept}
        
        % {\headfont \Large 学籍番号  \LARGE 著者名}
        % \vspace{10truept}
        % % 複数著者の場合は上2行を複製

        % \vspace{20truept}

        % {\headfont \Large 指導教員  指導教員名}
        % % 複数教員の場合、「\and」 でつなげる
        %%%%%%%%%%%%%%%%%%%%%%%%%%%%%

        %%%%% プロジェクト実習は以下 %%%%%
        {\headfont \Large 2024年度 修士論文}
        \vspace{20truept}

        {\headfont \Huge 津波避難誘導のマルチエージェント強化学習ドローンによるアプローチの検討}
        \vspace{160truept}
        
        {\headfont \Large 23VR008N  \LARGE 高林 秀}\\ \vspace{10truept}
        % 著者が増える場合は上行を複製
        
        \vspace{20truept}

        {\headfont \Large 指導教員  三宅陽一郎}
        % 複数教員の場合、「\and」 でつなげる
        %%%%%%%%%%%%%%%%%%%%%%%%%%%%%%%%
        
        \vspace{100truept}
        % ページがはみ出す場合はこの数字を調整

        {\headfont \Large 立教大学大学院}
        
        \vspace{10truept}
        
        {\headfont \Large 人工知能科学研究科 人工知能科学専攻}

    \end{center}
\end{titlepage}

% ===概要===
\begin{abstract}
    本研究は、津波避難誘導における津波避難ビルへの避難完了率の最大化を目的として、マルチエージェント強化学習を活用した自律飛行型ドローンモデルを提案する。
    観光地や都市部における避難者の多様性を考慮し、複数ドローンの協調行動により避難者を適切な避難所へ誘導する。
    特に、MA-POCAアルゴリズムを用いてエージェント間の協調性を強化し、ゲームAI技術や3D都市モデル技術を活用して現実の都市環境を再現したシミュレーション実験を実施した。
    実験は都市内の避難者を探索し誘導する群衆を形成する探索タスクと,群衆を避難所へ誘導する誘導タスクに分けて行い、比較実験によりその有効性を検証した。
    結果,今回作成したマルチエージェントモデルは,ルールベースで行動する場合より良い結果を得ることは出来なかったが,避難所誘導タスクにおいては,ルールベースで行動するドローンエージェントの誘導がない場合と比較して,避難完了率が向上することが確認された。
    
\end{abstract}

% ===目次===
\setcounter{tocdepth}{2}
 % 目次に表示する階層(必要に応じて変更)
 % 0はchapterまで、1でsectionまで、2でsubsectionまで...
\tableofcontents
% ===目次をページ数に入れない===
\thispagestyle{empty}
\clearpage
\addtocounter{page}{-1}

% ===本文(直書きでも可)===
% この方法ではFile outlineが出ない
\chapter{はじめに}
本章では,本論文の要旨および構成について述べる.
\section{要旨}
首都直下型地震や南海トラフ地震をはじめとする,大地震の 30 年以内の発生確率が 70\%~
80\%と非常に高くなっていることに加え,近年の豪雨など,将来の大規模災害のリスクが著
しく高まっている現状がある.
このような背景の下,本研究では,津波避難誘導のにおける避難完了率最適化と誘導人員のリスク軽減を目的として,マルチエージェント強化学習を活用した自律飛行型ドローンによる津波避難誘導モデルを提案する.
複数のドローンが協調して避難者を最適な避難所に誘導する方策を強化学習により求め,ゲームAIと3D都市モデルを用いた現実に近いシミュレーション環境による効果検証を実施した.
本提案手法により,津波避難誘導の課題解決に寄与することを目指す.

\section{本稿の構成}
まず,第2章において本稿の内容を理解するのに必要な事前知識,研究背景について述べる.
具体的には,以下の項目について説明する.
\begin{itemize}
  \item 強化学習についての基本説明
  \item マルチエージェントアルゴリズム MA-POCA(MultiAgent POsthumous Credit Assignment)について
  \item 本研究の社会的背景・課題について
\end{itemize}

次の第3章においては,提案手法の説明と本研究の研究方法についての説明を行う.
第4章では,マルチエージェント強化学習エージェントによる,津波避難誘導のシミュレーション実験の結果と考察を行う.
第5章では,実験結果をまとめ,マルチエージェントドローンによる津波避難誘導の実現可能性について論ずる.また,今後の研究の展望を述べる.

\chapter{研究背景}
本研究を理解する上で必要な概念である, 強化学習とそのアルゴリズムであるMA-POCAの理論や,関連する研究について述べる.
また,本研究を行うことになった社会的背景についても述べる.

\section{津波避難誘導における課題}
災害大国である我が国において,地震発生後の津波避難誘導オペレーションは非常に重要である.
特に近年,津波以外にも異常気象等による気象災害の激甚化もあり,避難誘導の遂行にあたって,益々その危険性も増していると推察される.\par 
本章では,我が国での津波避難誘導における課題について取り上げ,後述する提案手法の研究背景の理解を補助するものとする.
\subsection{オーバーツーリズムと観光地における避難誘導の課題}
近年の大幅な観光客増加と,観光地における避難誘導の課題,その関連性について述べる.
\paragraph{オーバーツーリズムとは}
\paragraph{観光地における避難誘導の課題}
\subsection{二次被害の発生}
津波避難誘導(あるいは,他の災害における避難誘導)においては,発災直後から二次被害にあう危険性が高い地域で活動しなければならないため,現場で誘導を行う警察や消防員等の安全確保が問題になっている.\par
\paragraph{風水害時における人的被害の特徴}
以下は,我が国で発生した1969年から2018年までの災害を対象に,消防団員が殉職した事例を消防白書や新聞記事,既往研究などから把握し,殉職時の状況を分析した結果が,山田らの研究\cite{yamada2020}によって報告されている.
\begin{quote}
  図-3 より,津波は,出動 途上,水防作業中,避難中,避難誘導中,人命救助中に 殉職者を出したことがわかった.なかでも避難誘導中と 避難中を合わせると全体で約 80\%を占めており,避難に関係する時に殉職者が出ている.
  \begin{figure}[H] 
    \centering 
    \includegraphics[width=0.8\textwidth]{Figures/fig-01.png}
    \caption{消防団員の災害フェーズ別殉職者の割合} 
    \label{fig:01} 
\end{figure}
\end{quote}
以上より,津波災害時の消防団員おけるの2次被害に関しては,避難誘導中が最も多い結果であることが示されている.上記は消防団員に限定した統計であるが,同じく避難誘導を行うすべての人員においても同様の傾向があると推察される.

  \subsection{災害時の必要人員不足}
\section{航空法改正によるドローンの災害対応における活用}
  \subsection{2022年12月5日の改正航空法の施行}
  \subsection{ドローンによる避難誘導の先行研究}
\section{強化学習}
%% TODO;機械学習と強化学習の基本的な枠組みについて述べる
  \subsection{MA-POCA(MultiAgent POsthumous Credit Assignment)}
  %% TODO: 協調学習:MA-POCAについての説明


\chapter{提案手法と実験概要}  
本章では、本研究が最終的に目指す津波避難誘導問題への解決策として、マルチエージェント強化学習と自律飛行型ドローンを組み合わせた提案手法について述べる。
また、提案手法が既存研究と異なる点や新規性についても論じる。

\section{提案手法の概要}  
本研究では、観光地や都市部といった地元住民以外にも多数の人々が屋外に存在する状況を想定している。
これには、日常的に避難訓練を受けていない観光客や土地勘のない訪問者も含まれる。
このような状況や、前章で述べた避難誘導における課題を背景に、地震発生後の津波避難という非常に緊急性の高い場面を想定し、避難誘導を行うための手法を検討する。
従来、自治体職員や警察・消防隊員といった人間が担ってきた避難誘導を、自律飛行型ドローンが代替するシステムを構築することを目標とする。

具体的には、マルチエージェント強化学習を活用し、複数のドローンエージェントが協調して行動する能力を学習させることで、刻々と変化する被災地域の状況を動的に認識し、群衆の避難完了率を最大化することを目指す。
また、避難者の位置、避難経路上の障害物、各ドローンの位置などをリアルタイムに反映するデジタルツイン環境を構築し、その環境内で学習済みのエージェントがシミュレーションを通じて最適な誘導方法を実行できるようにする。
さらに、デジタルツイン環境と実機ドローンを連動させることで、現実世界での運用を可能とする動的な避難誘導システムを構築する。

% TODO: ここに手法をまとめた図を貼る
\begin{figure}[H] 
  \centering 
  \includegraphics[width=1.0\textwidth]{Figures/2024-12-06 182816.png}
  \caption{提案手法 概略図} 
  \label{fig:01} 
\end{figure}

\section{既存研究との新規性}  
研究背景で述べたとおり、本研究は既存の研究といくつかの重要な相違点や新規性を有している。

まず、ドローンの防災活用はまだ研究が始まったばかりの新しい分野であり、本研究はその発展に寄与するものである。特に、複数のドローンを連携して運用するシステムに、AIや強化学習モデルを導入する点は、本研究の独自性を示す重要な要素である。

さらに、本研究では都市モデルを活用した訓練環境を構築し、デジタルツイン技術を通じて現実環境における運用を想定している。このように、現実世界での実用性を考慮した研究は、防災分野においても新しい試みである。

また、本研究は避難ビルの収容定員など、経路条件以外の要素を考慮した避難誘導モデルの作成にも取り組んでいる。従来の研究の多くが、避難者自身の行動最適化を通じて避難完了率の向上を目指しているのに対し、本研究では避難完了率を最適化できる避難誘導方策そのものを追求している点で特徴的である。

以上のような取り組みにより、本研究は現実世界での応用可能性を持つ動的な津波避難誘導システムの実現を目指している。


\section{シミュレーション環境構築}
  \subsection{避難所の選定}

  \subsection{避難者の行動フロー}
\section{エージェントモデル}
  \subsection{移動経路の計算方法}
  \subsection{観測情報}
  \subsection{報酬関数と行動評価}
  

\chapter{実験結果と考察}
本章では,前章で述べた提案手法の有効性を検証するために行ったシミュレーション実験の結果について報告する。
また,各実験結果に基づき,提案手法が持つ課題やその改善の可能性について考察する.

\section{避難者探索タスク実験}
\subsection{学習結果モデルの評価}

\section{避難所誘導タスクの結果}
\subsection{モデル学習結果}
以下は各都市におけるマルチエージェントモデルの学習過程のグラフである.エントロピーの推移,グループ報酬の推移,をそれぞれグラフで示す.
\begin{figure}[H]
  \centering
  % 1枚目の画像
  \begin{subfigure}{0.45\textwidth}
      \centering
      \includegraphics[width=\textwidth]{Figures/YokosukaModel-Result.png}
      \caption{横須賀市環境での訓練結果}
      \label{fig:image1}
  \end{subfigure}
  % 2枚目の画像
  \begin{subfigure}{0.45\textwidth}
      \centering
      \includegraphics[width=\textwidth]{Figures/NumazuModel-Result.png}
      \caption{沼津市環境での訓練結果}
      \label{fig:image2}
  \end{subfigure}
  \caption{}
  \label{fig:side_by_side}
\end{figure}
まずエントロピーの結果であるが,これはモデルの行動決定のランダムさを示す指標であり,この値が低い程行動出力がランダムでない,すなわちモデルの学習した方策が収束していると見なすことができる.
これを見ると,どちらの都市においてもエントロピーが学習が進むにつれ減少しており,学習によりモデルの行動が収束していることがわかる.
最終的な数値の大小としては,横須賀市での学習環境の方が沼津市に比べてエントロピーが低く,モデルの行動出力としては前者の環境の方が安定性があると言える.\par 
次にグループ報酬の推移について見ると,横須賀市の環境においては,エピソードの進行に伴いグループ報酬が減少してしまっている.
対して,沼津市の環境においては,横須賀市の環境よりもバラつきはあるものの,全体としては学習が進むにつれて微増しており,エントロピーの結果とも合わせると良い方向に,グループ報酬の最大化にむけて方策が収束していったことが分かる.\par
しかし,最終的なグループ報酬の値に着目すると,両方の環境において0.3から0.45程度の報酬しか得ておらず,訓練全体を通してあまり良い結果が得られていないことがわかる.
\subsection{実験結果}
本節では、避難所誘導タスクの実験結果を報告する。本実験では,マルチエージェントモデルとの比較実験と合わせて以下の3パターンにおける最終的な避難完了率や時間経過ごとの避難完了率の推移を元に,
訓練したマルチエージェントモデルモデルの有効性を評価する.
なお,シミュレーションの制限時間は,各都市ごとに異なるが,100秒毎に段階的に増やしていく形で設定し,記録した.
\begin{enumerate}[(a)]
  \item 避難者のみで避難行動を行う場合
  \item ルールベースで行動するエージェントモデルによる誘導
  \item 学習済みマルチエージェントモデルによる誘導
\end{enumerate}
\subsubsection{横須賀市のケース}
シミュレーション制限時間は1800秒から2400秒の間で,エピソード毎に100秒ずつ増加する形で検証した.
以下に,横須賀市のケースにおいて,各ケース毎に複数回シミュレートした結果の経過時間ごとの避難完了率の推移を示す.横軸がシミュレーション経過時間(秒)であり,縦軸が避難完了率である.
\begin{figure}[H]
  \centering
  % 上段左側
  \begin{minipage}{0.45\textwidth}
      \centering
      \includegraphics[width=\textwidth]{Figures/Yokosuka-EvaOnly-ERE.png} % 画像A
      \caption{(a).避難者のみで避難行動を行う場合}
      \label{fig:graph-a}
  \end{minipage}
  \hfill % 隙間を調整
  % 上段右側
  \begin{minipage}{0.45\textwidth}
      \centering
      \includegraphics[width=\textwidth]{Figures/Yokosuka-RuleModel-ERE.png} % 画像B
      \caption{(b).ルールベースでの誘導を導入時}
      \label{fig:graph-b}
  \end{minipage}
  
  % 改行して下段中央
  \vspace{1em} % 適宜調整
  \begin{minipage}{0.65\textwidth}
      \centering
      \includegraphics[width=\textwidth]{Figures/Yokosuka-AgentModel-ERE.png} % 画像C
      \caption{(c).学習済みマルチエージェントモデルによる誘導時}
      \label{fig:graph-c}
  \end{minipage}
\end{figure}
横須賀市における環境においては,(c)マルチエージェントモデルによる誘導よりも,(b)ルールベースまたは(a)避難者のみで避難行動を行う場合の方が避難完了率の変化が速く,多くのエピソードにおいて最終的な避難完了率が$90\%$を超えており,高いことがわかる.
マルチエージェントモデルによる誘導では,多くの場合で避難完了率が60\%から70\%程度で収束しており,多くのケースで避難者全員を制限時間以内に避難所まで誘導することを達成出来なかった.
また,ルールベースと避難者のみでの結果のグラフに着目すると,避難者のみで行動する場合は,避難率が100\%になるまでに要した時間が1500秒前後なのに対し,ルールベースでの誘導を導入した場合は1250秒前後と,出現した避難者全員が避難完了するまでの時間が数分程度短縮されていることがわかる.\par
推論時のエージェントの行動を観察すると,図の様に
\subsubsection{沼津市のケース}
シミュレーション制限時間は240秒から1800秒の間で,エピソード毎に100秒ずつ増加する形で検証した.
以下に,沼津市のケースにおいて,各ケース毎に複数回シミュレートした結果の経過時間ごとの避難完了率の推移を示す.横軸がシミュレーション経過時間(秒)であり,縦軸が避難完了率である.
\begin{figure}[H]
  \centering
  % 上段左側
  \begin{minipage}{0.45\textwidth}
      \centering
      \includegraphics[width=\textwidth]{Figures/Numazu-EvaOnly-ERE.png} % 画像A
      \caption{(a).避難者のみで避難行動を行う場合}
      \label{fig:graph-a}
  \end{minipage}
  \hfill % 隙間を調整
  % 上段右側
  \begin{minipage}{0.45\textwidth}
      \centering
      \includegraphics[width=\textwidth]{Figures/Numazu-RuleModel-ERE.png} % 画像B
      \caption{(b).ルールベースでの誘導を導入時}
      \label{fig:graph-b}
  \end{minipage}
  
  % 改行して下段中央
  \vspace{1em} % 適宜調整
  \begin{minipage}{0.65\textwidth}
      \centering
      \includegraphics[width=\textwidth]{Figures/Numazu-AgentModel-ERE.png} % 画像C
      \caption{(c).学習済みマルチエージェントモデルによる誘導時}
      \label{fig:graph-c}
  \end{minipage}
\end{figure}
沼津市のケースでは,(a)避難者のみで避難行動を行う場合と(b)ルールベースでの誘導を導入時の避難完了率の推移に大きな差は見られず,
どちらのケースも多くのエピソードで、短時間で避難完了率が90\%以上に達しており、全体的に迅速な避難が実現されている。
避難者のみで行動を行う場合,避難完了率が90\%を超えるまで開始から250秒以上経過してからが多いが,ルールベースでの誘導時は若干ではあるが250秒以下で避難完了率90\%を達成しているケースが多く,前者よりも避難完了率の上昇が早いことがわかる.
一方で,学習済みマルチエージェントモデルによる誘導時の避難完了率の推移を見ると,避難者のみで避難行動を行う場合やルールベースでの誘導を導入時と比較して,避難完了率の上昇が遅いことがわかる.
制限時間が短いエピソードでは避難完了率が20\%から40\%程に留まっている他,避難完了率が90\%を超えるまでに1200秒以上必要なエピソードが多く見られ,他のケースと比較して遅い避難完了率の推移が見られる.

\subsection{実験結果の考察}
避難所誘導タスクにおいて,図4.2から図4.7の結果から今回訓練したマルチエージェントモデルでは,多くの場合において避難完了率が90\%を超えることができず,避難者全員を制限時間内に避難所まで誘導することが難しいことがわかった.
また,経過時間あたりの避難完了率の伸び率からも,ルールベースでの誘導や避難者のみでの避難行動を行う場合と比較して,今回作成したマルチエージェントモデルによる誘導の方が遅いことが示された.
これは,図4.1の学習結果からもわかるように,モデルの学習が不十分であることが原因であると考えられる.
訓練全体を通じて,高い避難完了率を達成できるような方策をエージェントが経験できず,誘導人数や現在位置、避難所の収容人数に基づいた適切な誘導先の避難所を選択できていないことを示している.
また,エージェントの行動過程を分析すると,
\chapter{結果と考察}


% ===謝辞(任意)===
\chapter{謝辞}
ddd

% ===参考文献===
\bibliography{main}
\bibliographystyle{unsrt}
    

% ===付録(任意)===
\appendix
\chapter{付録A}
ddd


\end{document}
\documentclass{article}[jsarticle]
\usepackage[T1]{fontenc}
\usepackage[dvipdfmx]{hyperref}
\usepackage{lmodern}
\usepackage{latexsym}
\usepackage{amsfonts}
\usepackage{amssymb}
\usepackage{mathtools}
\usepackage{nccmath}
\usepackage{amsthm}
\usepackage{multirow}
\usepackage{graphicx}
\usepackage[dvipdfmx]{color}
\usepackage{wrapfig}
\usepackage{here}
\usepackage{float}
\usepackage{ascmac}
\usepackage{url}
\usepackage{listings}
\usepackage{xcolor}
\usepackage{pifont}

\lstset{
    basicstyle=\ttfamily\color{white},
    numbers=none,  % Line numbers
    numberstyle=\tiny\color{white},
    numbersep=5pt,
    tabsize=2,
    extendedchars=true,
    breaklines=true,
    keywordstyle=\color[rgb]{0.58,0.00,0.83},
    stringstyle=\color[rgb]{0.81,0.36,0.00},
    identifierstyle=\color{white},
    commentstyle=\color[rgb]{0.34,0.62,0.16},
    rulecolor=\color[rgb]{0.5,0.5,0.5},
    xleftmargin=0.1cm,    % Left margin
    xrightmargin=0.1cm,   % Right margin
    language=python,
    backgroundcolor=\color[rgb]{0.13,0.13,0.13},
    showspaces=false,
    showstringspaces=false
}



\title{特別研究3 研究報告書}
\author{高林秀 \\ 三宅研究室 博士前期課程2年 \\ V-CampusID : 23vr008n}
\date{\today}

\begin{document}

\maketitle

\begin{abstract}
    \noindent
    本稿は2024年度特別研究3の研究報告書である。前半は研究テーマの概要と説明を、後半は7月31日現在の研究進捗状況を報告するものである。\par
    \noindent
    特別研究3(以下、本研究と呼称)では、昨年度特別研究2まで行っていた「自律航行ドローンによる物資輸送アプローチの検討」から研究課題を変更し、
    『津波避難誘導の「マルチエージェント強化学習」とドローンによるアプローチの検討』とした。\par 
    \noindent
    % TODO本稿内容の要約
\end{abstract}

\tableofcontents

\section{研究課題について}
\subsection{概要説明}
\subsection{研究背景}
\subsection{新規性・最終目標}
\section{これまでの取り組み}
\subsection{調査関係}
\subsubsection{我が国における津波避難の課題についての調査}
\subsubsection{先行研究, 類似研究の調査}
\subsubsection{実機実験に向けた調査}
\subsection{シミュレーション・実験関係}
\subsubsection{簡易シミュレータの開発}
\subsubsection{都市モデルでのシミュレーション}
\section{現在の進捗状況}
\subsection{進捗状況まとめ}
\subsection{現在の課題}
\section{修士論文に向けて}
\appendix

\end{document}